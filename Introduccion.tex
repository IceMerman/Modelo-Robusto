\chapter{Introducción}\label{ch:introinfo}

%Introducción al problema y a las formulaciones
En los años noventas Colombia entró en un proceso de apertura y privatización de su economía. Se liberó el sector eléctrico y dando pie a una nueva estructura en   mercado eléctrico del país. En este proceso se crea un mercado mayorista donde se dan transacciones como: venta en la bolsa de energía, contratos a largo plazo, Transacciones Internacionales de Energía (TIE), contratos bilaterales, cargo por confiabilidad, entre otros elementos que al día de hoy hacen parte del mercado.

\noindent En Colombia \cite{rodas_gallego_evaluacion_2016}, la empresa Expertos en Mercados (XM) es la responsable de administrar el mercado eléctrico. Se encarga de definir la cantidad de energía a comprar a cada uno de los ofertantes (generadores) y el precio de esta energía; satisfaciendo criterios operativos y restricciones técnicas de los diferentes equipos que hacen parte del Sistema Interconectado Nacional (SIN). Las restricciones pueden incluir potencia mínimas generadas por planta, límites operativos de las líneas de transmisión, rampas de subida y bajada de planas y unidades, características y estados de las plantas,  entre otras.